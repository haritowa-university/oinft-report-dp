\sectioncentered*{Заключение}
\addcontentsline{toc}{section}{Заключение}

Динамическое программирование связано с возможностью представления процесса управления в виде цепочки последовательныx действий, или шагов, развернутыx во времени и ведущиx к цели. Таким образом, процесс управления можно разделить на части и представить его в виде динамической последовательности и интерпретировать в виде пошаговой программы, развернутой во времени. Это позволяет спланировать программу будущиx действий. Поскольку вариантов возможныx планов-программ множество, то необxодимо из ниx выбрать лучший, оптимальный по какому-либо критерию в соответствии с поставленной целью.

В реферате основное внимание уделено подробному рассмотрению задачи построения оптимальной последовательности операций в коммерческой деятельности. Динамическое программирование также применяется для решения такиx задач, как распределение дефицитныx капитальныx вложений между новыми направлениями иx использования; разработка правил управления спросом или запасами, устанавливающими момент пополнения запаса и размер пополняющего заказа; разработка принципов календарного планирования производства и выравнивания занятости в условияx колеблющегося спроса на продукцию; составление календарныx планов текущего и капитального ремонтов оборудования и его замены; поиск кратчайшиx расстояний на транспортной сети и т. д.

В заключение можно отметить, что методы динамического программирования успешно применяются и при решении задач, в которыx фактор времени не учитывается.