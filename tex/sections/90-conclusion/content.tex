\sectioncentered*{Заключение}
\addcontentsline{toc}{section}{Заключение}

Динамическое программирование связано с возможностью представления процесса управления в виде цепочки последовательных действий, или шагов, развернутых во времени и ведущих к цели. Таким образом, процесс управления можно разделить на части и представить его в виде динамической последовательности и интерпретировать в виде пошаговой программы, развернутой во времени. Это позволяет спланировать программу будущих действий. Поскольку вариантов возможных планов-программ множество, то необходимо из них выбрать лучший, оптимальный по какому-либо критерию в соответствии с поставленной целью.

В курсовой работе основное внимание уделено подробному рассмотрению задачи построения оптимальной последовательности операций в коммерческой деятельности. Динамическое программирование также применяется для решения таких задач, как распределение дефицитных капитальных вложений между новыми направлениями их использования; разработка правил управления спросом или запасами, устанавливающими момент пополнения запаса и размер пополняющего заказа; разработка принципов календарного планирования производства и выравнивания занятости в условиях колеблющегося спроса на продукцию; составление календарных планов текущего и капитального ремонтов оборудования и его замены; поиск кратчайших расстояний на транспортной сети и т. д.

В заключение можно отметить, что методы динамического программирования успешно применяются и при решении задач, в которых фактор времени не учитывается.