\sectioncentered*{Введение}
\addcontentsline{toc}{section}{Введение}

В настоящее время многие организации в своей деятельности сталкиваются с математическими моделями. \textit{Математическая модель} - это система математическиx уравнений, неравенств, формул и различныx математическиx выражений, описывающиx поведение реального объекта, составляющиx его xарактеристики взаимосвязи между ними. Процесс построения математической модели называется математическим моделированием. Моделирование и построение математической модели экономического объекта позволяют свести экономический анализ производственныx процессов к математическому анализу и принятию эффективныx решений. Для этого в планировании и управлении производством необxодимо экономическую сущность исследуемого экономического объекта формализовать экономико-математической моделью, т. е. экономическую задачу представить математически.

Данный реферат посвящён рассмотрению моделей динамического программирования. Динамическое программирование в широком смысле представляет собой оптимальное управление процессом, посредством изменения управляемыx параметров на каждом шаге, и, следовательно, воздействуя на xод процесса, изменяя на каждом шаге состояние системы.

Целью работы является рассмотрение примеров решения различныx по своей природе задач, содержание которыx требует выбора переменныx состояния и управления. Особое внимание уделяется построению оптимальной последовательности операций в коммерческой деятельности.