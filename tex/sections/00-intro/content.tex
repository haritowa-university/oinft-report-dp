\sectioncentered*{Введение}
\addcontentsline{toc}{section}{Введение}

В настоящее время многие организации в своей деятельности сталкиваются с математическими моделями. \textit{Математическая модель} - это система математических уравнений, неравенств, формул и различных математических выражений, описывающих поведение реального объекта, составляющих его характеристики взаимосвязи между ними. Процесс построения математической модели называется математическим моделированием. Моделирование и построение математической модели экономического объекта позволяют свести экономический анализ производственных процессов к математическому анализу и принятию эффективных решений. Для этого в планировании и управлении производством необходимо экономическую сущность исследуемого экономического объекта формализовать экономико-математической моделью, т. е. экономическую задачу представить математически.

Данный реферат посвящён рассмотрению моделей динамического программирования. Динамическое программирование в широком смысле представляет собой оптимальное управление процессом, посредством изменения управляемых параметров на каждом шаге, и, следовательно, воздействуя на ход процесса, изменяя на каждом шаге состояние системы.

Целью работы является рассмотрение примеров решения различных по своей природе задач, содержание которых требует выбора переменных состояния и управления. Особое внимание уделяется построению оптимальной последовательности операций в коммерческой деятельности.