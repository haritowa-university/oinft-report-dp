\subsection{Принцип оптимальности и математическое описание динамического процесса управления}

В основе метода динамического программирования лежит принцип оптимальности, впервые сформулированный в 1953 г. американским математиком Р. Э. Беллманом: каково бы ни было состояние системы в результате какого-либо числа шагов, на ближайшем шаге нужно выбирать управление так, чтобы оно в совокупности с оптимальным управлением на всех последующих шагах приводило к оптимальному выигрышу на всех оставшихся шагах, включая выигрыш на данном шаге. При решении задачи на каждом шаге выбирается управление, которое должно привести к оптимальному выигрышу. Если считать все шаги независимыми, тогда оптимальным управлением будет то управление, которое обеспечит максимальный выигрыш именно на данном шаге. Однако, например, при покупке новой техники взамен устаревшей на ее приобретение затрачиваются определенные средства, поэтому доход от ее эксплуатации в начале может быть небольшой, а в следующие годы новая техника будет приносить больший доход. И наоборот, если принято решение оставить старую технику для получения дохода в текущем году, то в дальнейшем это приведет к значительным убыткам. Этот пример демонстрирует следующий факт: в многошаговых процессах управление на каждом конкретном шаге надо выбирать с учетом его будущих воздействий на весь процесс.

Кроме того, при выборе управления на данном шаге следует учитывать возможные варианты состояния предыдущего шага. Например, при определении количества средств, вкладываемых в предприятие в $i$-м году, необходимо знать, сколько средств осталось в наличии к этому году и какой доход получен в предыдущем $(i-1)$-м году. Таким образом, при выборе шагового управления необходимо учитывать следующие требования:

\begin{enumerate}
	\item Возможные исходы предыдущего шага $S_{k-1}$.
	\item Влияние управления $х_k$ на все оставшиеся до конца процесса шаги $(n - k)$.
\end{enumerate}

В задачах динамического программирования первое требование учитывают, делая на каждом шаге условные предположения о возможных вариантах окончания предыдущего шага и проводя для каждого из вариантов условную оптимизацию. Выполнение второго требования обеспечивается тем, что в этих задачах условная оптимизация проводится от конца процесса к началу.

\textbf{Условная оптимизация}. На первом этапе решения задачи, называемом условной оптимизацией, определяются функция Беллмана и оптимальные управления для всех возможных состояний на каждом шаге, начиная с последнего в соответствии с алгоритмом обратной прогонки. На последнем, $n$-м шаге, оптимальное управление – $x^*_n$ определяется функцией Беллмана: $F(S) = \max \{W_n (S, x_n)\}$, в соответствии с которой максимум выбирается из всех возможных значений $x_n$, причем $x_n \in X$.

Дальнейшие вычисления производятся согласно рекуррентному соотношению, связывающему функцию Беллмана на каждом шаге с этой же функцией, но вычисленной на предыдущем шаге. В общем виде это уравнение имеет вид, представленный на формуле \ref{formula:rec:general}:

\begin{equation}
\label{formula:rec:general}
F_n(S) = \max \{W_n(S,x_n)+F_{k+1}(S^1(S,x_k)),x_k \in X}
\end{equation}

Этот максимум (или минимум) определяется по всем возможным для $k$ и $S$ значениям переменной управления $X$.

\textbf{Безусловная оптимизация}. После того, как функция Беллмана и соответствующие оптимальные управления найдены для всех шагов с $n$-го по первый, осуществляется второй этап решения задачи, называемый безусловной оптимизацией. Пользуясь тем, что на первом шаге $(k = 1)$ состояние системы известно – это ее начальное состояние $S_o$, можно найти оптимальный результат за все $n$ шагов и оптимальное управление на первом шаге x1, которое этот результат доставляет. После применения этого управления система перейдет в другое состояние $S_1(S,х^*_1)$, зная которое, можно, пользуясь результатами условной оптимизации, найти оптимальное управление на втором шаге $х^*_2$, и так далее до последнего $n$-го шага. Вычислительную схему динамического программирования можно строить на сетевых моделях, а также по алгоритмам прямой прогонки (от начала) и обратной прогонки (от конца к началу). Рассмотрим примеры решения различных по своей природе задач, содержание которых требует выбора переменных состояния и управления.