\subsubsection{Постановка задачи динамического программирования}

Постановку задачи динамического программирования рассмотрим на примере инвестирования, связанного с распределением средств между предприятиями. В результате управления инвестициями система последовательно переводится из начального состояния $S_0$ в конечное $S_n$. Предположим, что управление можно разбить на n шагов и решение принимается последовательно на каждом шаге, а управление представляет собой совокупность n пошаговых управлений. На каждом шаге необходимо определить два типа переменных: переменную состояния системы $S_k$ и переменную управления $x_k$. Переменная $S_k$ определяет, в каких состояниях может оказаться система на рассматриваемом $k$-м шаге. В зависимости от состояния $S$ на этом шаге можно применить некоторые управления, которые характеризуются переменной $x_k$, которые удовлетворяют определенным ограничениям и называются допустимыми.

Допустим, $X = (x_1, x_2, \cdots, x_k, \cdots, x_n)$ – управление, переводящее систему из состояния $S_0$ в состояние $S_n$, a $S_k$ – есть состояние системы на $k$-м шаге управления. Тогда последовательность состояний системы можно представить в виде графа, изображенного на рис. 1.

\begin{figure}[h]
  \centering
    $(1+2+3+a=c)$
  \caption{Схема общения при помощи MTProto}
  \label{sec:analysis:research:analogs:telegram:mtproto1}
\end{figure}