\subsection{Постановка задачи динамического программирования}

Постановку задачи динамического программирования рассмотрим на примере инвестирования, связанного с распределением средств между предприятиями. В результате управления инвестициями система последовательно переводится из начального состояния $S_0$ в конечное $S_n$. Предположим, что управление можно разбить на n шагов и решение принимается последовательно на каждом шаге, а управление представляет собой совокупность n пошаговых управлений. На каждом шаге необходимо определить два типа переменных: переменную состояния системы $S_k$ и переменную управления $x_k$. Переменная $S_k$ определяет, в каких состояниях может оказаться система на рассматриваемом $k$-м шаге. В зависимости от состояния $S$ на этом шаге можно применить некоторые управления, которые характеризуются переменной $x_k$, которые удовлетворяют определенным ограничениям и называются допустимыми.

Допустим, $X = (x_1, x_2, \textellipsis, x_k, \textellipsis, x_n)$ – управление, переводящее систему из состояния $S_0$ в состояние $S_n$, a $S_k$ – есть состояние системы на $k$-м шаге управления. Тогда последовательность состояний системы можно представить в виде графа, изображенного на Рисунке \ref{sec:task:definition:system-state}.

\begin{figure}[h]
  \centering
    $S_0 \rightarrow S_1 \rightarrow \textellipsis \rightarrow S_{k-1} \rightarrow S_k \rightarrow \textellipsis \rightarrow Sn$
  \caption{График состояний системы}
  \label{sec:task:definition:system-state}
\end{figure}

Применение управляющего воздействия $x_k$ на каждом шаге переводит систему в новое состояние $S^1(S, x_k)$ и приносит некоторый результат $W_k (S, x_k)$. Для каждого возможного состояния на каждом шаге среди всех возможных управлений выбирается оптимальное управление $х^*_k$, такое, чтобы результат, который достигается за шаги с $k$-го по последний $n$-й, оказался бы оптимальным. Числовая характеристика этого результата называется функцией Беллмана $F_k(S)$ и зависит от номера шага $k$ и состояния системы $S$.

Задача динамического программирования формулируется следующим образом: требуется определить такое управление $X^*$, переводящее систему из начального состояния $S_0$ в конечное состояние $S_n$, при котором целевая функция принимает наибольшее (наименьшее) значение $F(S_0, X^*) \rightarrow \text{extr}$.

Особенности математической модели динамического программирования заключаются в следующем:
\begin{enumerate}
	\item Задача оптимизации формулируется как конечный многошаговый процесс управления.
	\item Целевая функция (выигрыш) является аддитивной и равна сумме целевых функций каждого шага, что показано в формуле \ref{formula:sum:ext}.
		\begin{equation}
		\label{formula:sum:ext}
		F = \sum_{k=1}{F_k(S_{k-1},x_k)} \rightarrow extremum
		\end{equation}
	\item Выбор управления $х_k$ на каждом шаге зависит только от состояния системы к этому шагу $S_{k−1}$, и не влияет на предшествующие шаги (нет обратной связи).
	\item Состояние системы $S_k$ после каждого шага управления зависит только от предшествующего состояния системы $S_{k-1}$ и этого управляющего воздействия $х_k$ (отсутствие последействия) и может быть записано в виде уравнения состояния: $S_k = f_k (S_{k-1}, х_k), k = 1, n$.
	\item На каждом шаге управление $х_k$ зависит от конечного числа управляющих переменных, а состояние системы $S_k$ зависит от конечного числа параметров.
	\item Оптимальное управление представляет собой вектор $X^*$, определяемый последовательностью оптимальных пошаговых управлений: $X = (х^*_1, х^*_2, \textellipsis, х^*_k, \textellipsis, х^*_n), число которых и определяет количество шагов задачи.
\end{enumerate}