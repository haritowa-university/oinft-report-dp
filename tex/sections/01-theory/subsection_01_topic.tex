\subsubsection{Предмет динамического программирования}

Динамическое программирование представляет собой математический аппарат, который подходит к решению некоторого класса задач путем их разложения на части, небольшие и менее сложные задачи. При этом отличительной особенностью является решение задач по этапам, через фиксированные интервалы, промежутки времени, что и определило появление термина динамическое программирование. Следует заметить, что методы динамического программирования успешно применяются и при решении задач, в которых фактор времени не учитывается. В целом математический аппарат можно представить как пошаговое или поэтапное программирование. Решение задач методами динамического программирования проводится на основе сформулированного Р. Э. Беллманом принципа оптимальности: оптимальное поведение обладает тем свойством, что каким бы ни было первоначальное состояние системы и первоначальное решение, последующее решение должно определять оптимальное поведение относительно состояния, полученного в результате первоначального решения.

Из этого следует, что планирование каждого шага должно проводиться с учетом общей выгоды, получаемой по завершении всего процесса, что и позволяет оптимизировать конечный результат по выбранному критерию.

Таким образом, динамическое программирование в широком смысле представляет собой оптимальное управление процессом, посредством изменения управляемых параметров на каждом шаге, и, следовательно, воздействуя на ход процесса, изменяя на каждом шаге состояние системы.

В целом динамическое программирование представляет собой стройную теорию для восприятия и достаточно простую для применения в коммерческой деятельности при решении как линейных, так и нелинейных задач.

Динамическое программирование является одним из разделов оптимального программирования. Для него характерны специфические методы и приемы, применительные к операциям, в которых процесс принятия решения разбит на этапы (шаги). Методами динамического программирования решаются вариантные оптимизационные задачи с заданными критериями оптимальности, с определенными связями между переменными и целевой функцией, выраженными системой уравнений или неравенств. При этом, как и в задачах, решаемых методами линейного программирования, ограничения могут быть даны в виде равенств или неравенств. Однако если в задачах линейного программирования зависимости между критериальной функцией и переменными обязательно линейны, то в задачах динамического программирования эти зависимости могут иметь еще и нелинейный характер. Динамическое программирование можно использовать как для решения задач, связанных с динамикой процесса или системы, так и для статических задач, связанных, например, с распределением ресурсов. Это значительно расширяет область применения динамического программирования для решения задач управления. А возможность упрощения процесса решения, которая достигается за счет ограничения области и количества, исследуемых при переходе к очередному этапу вариантов, увеличивает достоинства этого комплекса методов.

Вместе с тем динамическому программированию свойственны и недостатки. Прежде всего, в нем нет единого универсального метода решения. Практически каждая задача, решаемая этим методом, характеризуется своими особенностями и требует проведения поиска наиболее приемлемой совокупности методов для ее решения. Кроме того, большие объемы и трудоемкость решения многошаговых задач, имеющих множество состояний, приводят к необходимости отбора задач малой размерности либо использования сжатой информации. Последнее достигается с помощью методов анализа вариантов и переработки списка состояний.
Для процессов с непрерывным временем динамическое программирование рассматривается как предельный вариант дискретной схемы решения. Получаемые при этом результаты практически совпадают с теми, которые получаются методами максимума Л. С. Понтрягина или Гамильтона-Якоби-Беллмана.

Динамическое программирование применяется для решения задач, в которых поиск оптимума возможен при поэтапном подходе, например, распределение дефицитных капитальных вложений между новыми направлениями их использования; разработка правил управления спросом или запасами, устанавливающими момент пополнения запаса и размер пополняющего заказа; разработка принципов календарного планирования производства и выравнивания занятости в условиях колеблющегося спроса на продукцию; составление календарных планов текущего и капитального ремонтов оборудования и его замены; поиск кратчайших расстояний на транспортной сети; формирование последовательности развития коммерческой операции и т.д.